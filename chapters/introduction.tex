\chapter{Introduction}
% \addcontentsline{toc}{chapter}{Introduction}

In the modern day, fingerprint recognition has become a crucial tool for biometric security, authentication, and personal identification. Biometric security systems use fingerprint recognition for its reliability and uniqueness towards fingertip ridge patterns, making them ideal for verifying identification over traditional authentication methods such as passwords and tokens, which are susceptible to loss, theft, and memory loss. The early works of \textcite{PersonalIdentificationDescription} laid the foundation of scientific study into fingerprints and their use in forensics and personal identification by introducing Level 2 features, such as minutiae points or where fingerprint ridges end or split \parencite{PersonalIdentificationDescription}. Galton discovered the existence of sweat pores on ridges; however, he did not suggest a method for their use in identification. It wasn't until 1912 that Locard introduced "poroscopy," or the comparison of sweat pores for personal identification based on their size, shape, position on the ridge, and frequency \parencite{locardLidentificationCriminelsPar1912}. In 1962, Chatterjee introduced "edgeoscopy," or the use of ridge edge shapes with other fingerprint features for individualization based on eight categories: straight, convex, peak, table, pocket, concave, angle, and others \parencite{ashbaughQuantitativequalitativeFrictionRidge1999}. The use of permanent and unique Level 3 features, such as sweat pores and ridge edge curvature, enhances minutiae-based identification.

Fingerprints have been shown to have low False Accept Rates (FAR) and False Reject Rates (FRR), making them renowned for their universality, distinction, and fast performance, leading to the creation of large-scale databases such as the FBI's over 200 million fingerprint records \parencite{chengArtificialFingerprintRecognition2006}. Automated Fingerprint Identification Systems (AFIS) have evolved into reliable systems for biometric matching, specifically in criminal investigation and various other law enforcement domains. However, conventional AFIS relies on Level 1 features, such as ridge structure and pattern, and Level 2 features. This form of 2D level surface imaging, while effective, introduces limitations that affect the robustness and security of common fingerprint recognition systems. 

One major challenge of using the outermost surface skin of the fingerprint, called the stratum corneum, is that it is susceptible to damage, skin conditions, and distortion during scanning due to the skin's elasticity or incomplete contact. Skin conditions, such as whether the skin is wet, dry, or stained during image acquisition, can lead to obscured fingerprint ridge valley patterns and overall incorrect representation of fingerprint features. Furthermore, these 2D scanners leave them highly vulnerable to presentation attacks such as spoofing, where artificial fingerprints are created using various resources such as liquid silicone rubber, gelatin, and latex. Many studies have shown that original fingerprints can be cloned and reconstructed using minutiae points by attackers and stored in user templates \parencite{sunSynchronousFingerprintAcquisition2020,liuFingerprintPresentationAttack2022,darlowEfficientInternalSurface2015}. Since the fingerprint is permanent and cannot be altered like a password, attackers collecting biometric information to breach databases is a significant issue. These limitations and underlying issues present a need for more advanced fingerprint recognition systems that look beyond surface-level imaging.

Optical Coherence Tomography (OCT) is a helpful technique that has transformed fingerprint analysis by providing a non-invasive, high-resolution 3D volumetric data of the fingertip's surface and sub-surface anatomical features. OCT systems, originally invented in medical fields such as ophthalmology and dermatology \parencite{fujimotoOpticalCoherenceTomography2000}, produce cross-sectional images called “B-scans” and depth profiles called “A-scans” to get a depiction of the multiple layers of the fingertip: the stratum corneum, viable epidermis, and papillary dermis.

Using OCT in fingerprint recognition helps address the limitations of traditional systems by extracting deeper anatomical regions in order to have more robust, secure, and accurate biometric matching. Deeper anatomical regions such as the internal fingerprint, which is formed at the viable epidermis junction, is referred to as the mother template since it is protected from outside surface injuries, deterioration from age, and environmental factors \parencite{yuMethodsApplicationsFingertip2023,dingSurfaceInternalFingerprint2021,liuLightweightNoiseRobustMethod2024}. The lack of complicated underlying biological structures in artificial fingerprints provides strong support against spoofing. OCT goes beyond traditional Level 1 and Level 2 minutiae by capturing Level 3 features such as sweat pores and ridge curvature. Sweat pores are known for their durability, consistency, and uniqueness, and align with fingerprint ridge patterns, showing potential as reliable biometric features, an idea established by Locard's poroscopy. Merging Level 3 features has shown significant improvement with performance, such as a 20 percent reduction in Equal Error Rate (EER) when combined with Level 1 and 2 features \parencite{jainPoresRidgesHighResolution2007}.
\section{Background and Related Work}
\subsection{Levels of Fingerprint}

Fingerprint analysis uses a three-level classification system with features categorized as Level 1, Level 2, and Level 3. This hierarchy is based on anatomical structure, with each level offering increasingly detailed information and serving a unique role in biometric recognition.

Level 1 fingerprint features refer to the macroscopic patterns visible on a fingerprint, such as ridge flow, whorls, loops, and arches. These features provide raw structural information for large classification instead of individual identification \parencite{maioDirectGrayscaleMinutiae1997}. For example, the Henry Classification System was developed in the late 19th century as the basis for fingerprint identification that depended on Level 1 patterns to organize a large collection of fingerprints by reducing and sorting the search space for forensic investigations (Henry, 1928).

As previously mentioned, Level 2 features add more detail by focusing on minutiae, which is where a ridge splits into two or ends abruptly. The splitting of a ridge is known as ridge bifurcations, and the abrupt ending of a ridge is known as ridge endings. They are used in modern biometric systems due to their stability and distinction for most tasks in identification \parencite{jainPoresRidgesHighResolution2007,maltoni2009handbook}. These minutiae points form the foundation for most automated fingerprint identification algorithms due to their spatial relationships and orientation to provide a framework for distinguishing individuals. The distinct appearance of these points across an individual's fingers makes them important for accurate and reliable biometric authentication.

Level 3 features offer the most detailed fingerprint information, including sweat pores, ridge contours, and underdeveloped ridges between the main ridges. Capturing these details requires high-resolution imaging, and research has shown that adding sweat pore data enhances identification accuracy, especially in forensic scenarios where only partial fingerprints are available \parencite{zhaoAdaptiveFingerprintPore2010}. Advances in fingerprint sensors and imaging technology now enable the capture of small anatomical features, boosting their use in biometric systems. This indicates that Level 3 features are crucial for high-security systems like border control checkpoints \parencite{liuOneClassFingerprintPresentation2021,sunNewApproachAutomated2023, zhangUniformRepresentationModel2024}. However, implementing Level 3-based systems remains challenging due to the high imaging resolution needed to reliably capture complex details with feature extraction algorithms \parencite{zhangSweatGlandExtraction2023,donidalabatiNovelPoreExtraction2018}.
\subsection{Prior Methods for Sweat Duct Detection}

Sweat ducts have become a popular subject of interest in advanced biometric research due to their uniqueness and stability over an individual's lifetime. Accurately detecting these ducts has been explored using a variety of methodologies involving traditional methods and deep learning techniques.

Traditional methods for detecting pores typically follow a carefully applied sequence of filtering and morphological operations on high-resolution fingerprint images to determine pore locations. \textcite{jainPoresRidgesHighResolution2007} introduced one of the earliest approaches that used Gabor filters and Mexican hat wavelet transform to enhance ridge contrast and intensify pore locations. While this method showed effectiveness for 1000 dpi resolution fingerprint images, it was tested on a small dataset, limiting its generalizability. \textcite{zhaoAdaptiveFingerprintPore2010} proposed a method to divide the fingerprint into blocks, estimate an adaptive ridge orientation in each block, and apply a match filter to accurately extract pore locations. Its adaptability boosted its accuracy over Jain; however, it struggles with dealing with distortion. \textcite{teixeiraImprovingPoreExtraction2014} introduced spatial filtering to measure distances between neighboring pores in a block and remove false positives. While it could accurately detect true pores, it struggles with determining false positives. These traditional methods, and many others, laid the foundation for sweat pore detection. Their sensitivity to noise, limited generalizability, and reliance on manual features have motivated the development of deep learning based techniques.

\textcite{donidalabatiNovelPoreExtraction2018} introduced a convolutional neural network (CNN) to detect pores using a pore intensity map with a threshold to locate pores. Their approach demonstrated that CNNs can potentially detect pores, but their capacity was limited for learning the complex patterns in fingerprint images with only two convolutional layers. \textcite{jangDeepPoreFingerprintPore2017} proposed DeepPore, a CNN with ten convolutional layers that generate pore intensity maps with spatial filtering applied to improve accuracy. The authors noted that it was trained on a small dataset, which could affect its scalability to other datasets. \textcite{anandPoreDetectionHighresolution2019} developed DeepResPore, an eighteen-layer residual learning CNN with eight residual blocks trained on over 200,000 fingerprint patch images. They tested it on two benchmark datasets and achieved state-of-the-art detection accuracy; however, its drawback was increased computational time. \textcite{sunSynchronousFingerprintAcquisition2020} created a fingerprint acquisition system combining Total Internal Reflection (TIR) and OCT to capture both surface and subsurface fingerprint structures, providing clearer visuals of sweat pores. While their work did not involve deep learning or pore detection, their technique and high-resolution data contributed to future models. The system's complexity and cost pose challenges for adoption outside research settings.

\textcite{yuNewApproachExternal2020} tackled the challenge of internal and external fingerprint registration by developing a method to minimize multisensor differences. Their approach significantly improved the alignment of external ridges with internal structures using OCT and supported accurate training models through high-quality registered data. A limitation was the difficulty in balancing resolution and acquisition speed for real-time applications. \textcite{liuRobustHighsecurityFingerprint2020} introduced a 2D recognition system that uses OCT to analyze Level 3 features primarily for anti-spoofing. They acknowledged the potential for integrating OCT-based features into learning-based classifiers. Their limitation was reliance on specialized and potentially inaccessible OCT hardware. \textcite{dingSubcutaneousSweatPore2021} presented a method using OCT to locate sweat pore positions with image analysis techniques to trace sweat duct pathways from the dermis to the fingerprint surface. Their approach offered insights into pore structure and could enhance neural network training data. However, susceptibility to image noise in deeper tissue layers affected overall accuracy. \textcite{liuNovelHighResolutionFingerprint2022} designed an approach to preserve internal and external features while maintaining detail, leading to more effective pore detection and feature extraction. Their key limitation was the trade-off between high image resolution and adequate acquisition speed in practical systems. Lastly, \textcite{sunNewApproachAutomated2023} reviewed methods and applications of fingertip biometrics using OCT. They highlighted various strategies for capturing and analyzing internal fingerprint features, emphasizing the importance of multimodal and deep learning approaches, and discussed challenges like the lack of standardized datasets and the need for efficient real-time fingerprint reconstruction systems.
\subsection{Challenges and Gaps in Current Research}
Although major advancements in fingerprint recognition systems have occurred, several important gaps remain in the research field. A key challenge hindering OCT-based fingerprint recognition is inconsistency across different devices. Recognition models often struggle to work reliably with data from new OCT machines due to small differences in imaging properties. This lack of interoperability means that these models are not easily generalizable. To make these systems practical for real-world applications, better adaptation techniques and larger datasets covering a wide range of devices are necessary. 

The inability to generalize to new threats because of poor dataset diversity represents a significant gap in Presentation Attack Detection (PAD). Many PAD methods' performance rates are limited against the constantly evolving landscape of spoofing techniques because they are frequently trained on small, homogeneous collections of spoof materials \parencite{liuFingerprintPresentationAttack2022,sunNewApproachAutomated2023,liuOneClassFingerprintPresentation2021}. It is essential to develop broader datasets that include various presentation attack instruments. This will not only improve detection accuracy for known attacks but also facilitate semi-supervised or unsupervised approaches capable of identifying spoofs they have not encountered before. Additionally, current AFIS systems lack effective integration of internal and external fingerprint features. While OCT allows imaging of internal structures such as the VE, there are no standardized protocols or efficient fusion algorithms for combining this internal data with traditional surface-level information. Merging internal and external features could significantly enhance recognition accuracy and spoof resistance; however, this area remains largely underexplored. Finally, although Level 3 features have shown promise, their practical implementation is hindered by the lack of longitudinal studies, standardized evaluation metrics, and scalable 3D processing frameworks. Future efforts in fingerprint recognition should focus on developing more advanced, adaptable, and efficient systems that surpass traditional surface imaging. This is crucial for addressing current limitations and ensuring reliable, secure, and scalable biometric authentication across diverse environments.

This paper is driven by the need to address the limitations of traditional 2D surface fingerprint systems to improve the accuracy of AFIS by using stable and secure subsurface data, such as sweat ducts and internal fingerprint ridge curvature, obtained through OCT. Using Level 3 features offers promising biometric markers that are less susceptible to distortion, damage, and spoofing. The primary objective of this work is to develop, to the best of our knowledge, the first 3D pipeline based on OCT for the extraction and analysis of 3D ridge structures and sweat ducts. Our study hypothesizes that sweat duct locations on the internal fingerprint surface are not random but follow distinctive spatial patterns, occurring at consistent distances along ridge structures, on curved ridge surfaces, and with local surface orientations that align between neighboring ducts. Our pipeline aims to address whether OCT-extracted Level 3 features provide anatomically stable and secure biometric markers for improving identification systems beyond traditional surface-level methods. 

The dissertation is structured as follows: Chapter 2 presents the proposed method; Chapter 3 presents the results; and Chapter 4 discusses the interpretation of results, limitations, and potential future directions.