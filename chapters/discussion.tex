\chapter{Discussion}
\subsection{Interpretation of Results}

This work aimed to investigate whether 3D OCT data could be used to extract and analyze Level 3 fingerprint features, specifically sweat ducts and internal ridge geometry, in a manner that could enhance biometric security systems. Traditional fingerprint recognition systems primarily rely on Level 1 and Level 2 features. These systems are limited by their dependence on surface imaging. The findings of this project provide initial evidence that internal anatomical features obtained from OCT scans offer distinct patterns. These patterns could serve as stable and spoof-resistant biometric markers. 

\begin{table}[ht]
\centering
\caption{Summary of quantitative results for the full dataset and subset of sweat duct points.}
\label{tab:results}
\resizebox{\textwidth}{!}{%
\begin{tabular}{lcccccc}
\hline
\textbf{Measure} & \textbf{Dataset} & \textbf{Mean} & \textbf{SD} & \textbf{t} & \textbf{p} & \textbf{Null Hypothesis} \\
\hline
Nearest-Neighbor Distance (mm) & Full & 0.1259 & 0.0444 & -- & -- & Not tested \\
                               & Subset & 0.1696 & 0.0679 & -- & -- & Not tested \\
Mean Curvature (mm$^{-1}$)     & Full & 5.6783 & 51.9530 & 3.2108 & 1.37 $\times$ 10$^{-3}$ & Rejected \\
                               & Subset & 4.7877 & 31.4448 & 1.5073 & 1.35 $\times$ 10$^{-1}$ & Not rejected \\
Normal Vector Dot Product      & Full & 0.7891 & 0.1995 & -29.2802 & 4.20 $\times$ 10$^{-127}$ & Rejected \\
                               & Subset & 0.8553 & 0.1458 & -9.2552 & 1.35 $\times$ 10$^{-14}$ & Rejected \\
\hline
\end{tabular}%
}
\end{table}

The quantitative results from the spatial distribution analysis show that sweat ducts are not randomly dispersed across the internal fingerprint surface. The mean nearest-neighbor distance across all sweat ducts was about 0.126 mm, with a relatively small standard deviation. This consistency indicates a stable spacing pattern. The finding aligns with prior anatomical studies that suggest sweat ducts follow the course of ridges and appear at relatively uniform intervals. Subset analysis, which examined ten sweat ducts, showed a slightly higher mean distance of 0.170 mm. This suggests that smaller samples reflect local variation while the overall population maintains a stable pattern. Such regularity is valuable for biometric recognition. It means sweat duct arrangements may be predictable enough to help matching algorithms, but still unique enough between individuals to serve as identifying characteristics. The curvature analysis further supports the potential of these features. Sweat duct points were found in regions of the ridge that are not perfectly flat. Statistical testing confirmed a significant deviation from zero curvature. This shows that sweat ducts tend to be located along ridge parts that have slight 3D contours. From a practical perspective, this curvature data adds spatial context for matching. It allows algorithms to encode not only the x and y positions of sweat ducts, but also their orientation on a curved surface. Having this depth of information could make it more difficult for attackers to create convincing spoofing attacks. A counterfeit fingerprint would need to replicate both curvature distribution and sweat duct spacing.

The normal vector analysis provides additional insight by examining how sweat duct orientations align. An average dot product of about 0.79 shows that most sweat ducts have similar directions compared to their neighbors. Some variation still exists across the fingerprint. The subset analysis showed a higher mean alignment of 0.86. This suggests that smaller fingerprint regions are more uniform in topography. Anatomically, this means that duct emergence angles are influenced by the microstructure of the ridge in a local area, while still exhibiting subtle changes along the ridge. These findings are significant because they reveal a multi-dimensional set of parameters: spacing, curvature, and orientation. These could be integrated into more advanced fingerprint matching algorithms. 

Overall, these results demonstrate that the internal fingerprint surface displays measurable and distinctive patterns in sweat duct distribution, surpassing the capabilities of traditional surface imaging. Using depth-resolved data with OCT, it becomes possible to develop richer biometric templates. These could combine Level 3 features with spatial and geometric information. This approach could make fingerprint recognition more robust. It is especially useful when surface ridges are degraded or intentionally altered, and it can improve resistance to presentation attacks.
\subsection{Relation to Previous Work}

The results extend earlier research on sweat pore and duct detection. This study progresses from 2D analysis to comprehensive 3D characterization. Traditional pore detection methods, such as Gabor filters or morphological processing, have been limited by imaging resolution and surface distortion. Although deep learning techniques have improved detection accuracy on high-resolution surface images, they still rely on 2D representations. By directly working with volumetric OCT data, this study captures the entry points of sweat ducts on the internal ridge surface. It also captures their 3D spatial relationships. 

The observed spacing patterns support Locard’s early claim that pores are stable in their location relative to ridges. However, this study adds quantitative curvature and orientation data, which classical poroscopy did not consider. Past OCT research has primarily focused on visualizing sweat ducts or aligning internal and external features for matching purposes. Here, the work advances the field by performing statistical analyses that describe the spatial organization of these features and how their geometry can be measured consistently. 

The findings also match recent studies showing Level 3 features can improve EER when combined with Level 1 and 2 features. For example, \textcite{jainPoresRidgesHighResolution2007} reported a 20 percent reduction in EER when pores were incorporated into recognition systems. The current results suggest that adding curvature and orientation parameters may further enhance matching performance. These parameters provide geometric constraints that are more difficult to spoof. The pipeline also offers a framework to address a key challenge in OCT fingerprint research: integrating internal and external features into a unified template \parencite{aliRobustBiometricAuthentication2020}. By segmenting and analyzing sweat ducts systematically, it becomes feasible to develop matching algorithms that can use internal features alone or combined with surface data.
\subsection{Limitations}

While these findings are promising, several limitations need to be acknowledged. The most immediate is the sample size. Although the dataset features high-resolution imaging, it includes a relatively small number of subjects and scanned regions. This limits the ability to generalize the results to a broader population. Variations in sweat duct spacing, curvature, and orientation may be influenced by factors such as age, sex, ethnicity, or skin condition. None of these were controlled in this dataset. 

Additionally, the use of a single OCT system for data collection. OCT devices differ in imaging depth, resolution, and noise levels. Models or thresholds developed from one system might not transfer well to different devices without recalibration. This limited cross-device applicability is a challenge in biometric OCT research. 

Furthermore, the manual selection of ROI’s and the splitting of anatomical planes. Interactive cropping and segmentation ensured anatomical accuracy for this proof-of-concept. However, these steps are impractical for large-scale or real-time use. Automated algorithms for region and mesh segmentation will be necessary to make this scalable. The analysis used assumptions that could potentially influence the results. For example, curvature and orientation were derived from the triangulated mesh, which depends on mesh resolution and smoothing. Small changes during preprocessing could alter numerical values. DBSCAN effectively clustered duct points and filtered noise, but parameter choices could change which sweat ducts are included. 

Finally, environmental and physiological factors may also influence the findings. Although the internal fingerprint is more stable than the surface, OCT imaging can still be affected by hydration, pressure, and finger micro-movements. Imaging conditions in this study were carefully controlled, but real-world environments may not be as consistent.
\subsection{Future Directions}

Future research should address these limitations and expand the scope of the analysis. Increasing the diversity and size of the dataset would enable more robust statistical modeling of sweat duct patterns and might reveal demographic influences. Long-term studies are needed to confirm the stability of internal duct features over months or years. From a technical standpoint, improving the interoperability of OCT-based biometric systems is essential. This will require collecting multi-device datasets and developing calibration methods that normalize geometric measurements across systems. Such standardization would ensure that templates generated on one device could be matched reliably against those from another.

Automating the segmentation and analysis pipeline is another priority. Machine learning models could be trained to detect ROI’s directly from volumetric data. This would eliminate the need for manual cropping. Similarly, algorithms could be created to automatically align the internal fingerprint surface to a standard coordinate system before curvature and orientation analysis. Integrating these internal features into operational recognition systems will require new template formats and matching algorithms. Instead of using just the x-y positions of sweat ducts, templates could encode each duct’s spacing, curvature context, and orientation vector. Matching algorithms could then compare these multi-parameter descriptors between scans. This could increase both accuracy and spoof resistance.

Another promising direction is multimodal fusion. OCT-derived internal features could be combined with traditional surface scans to create composite templates. Such fusion could be helpful when one modality is degraded, for example, when the surface is damaged but the internal structure is intact. Similarly, combining internal ridge data with external minutiae could provide redundancy against spoofing attempts. There is also an opportunity to investigate how these internal features perform in PAD. Artificial fingerprints usually lack accurate subsurface anatomy. Incorporating curvature and orientation checks could flag potential spoofs even if surface minutiae match. Future work should test this idea with a range of spoof materials and fabrication techniques. 

Ultimately, the biological and forensic implications of this work warrant further exploration. Understanding how sweat duct patterns relate to other skin microstructures could provide insights into developmental biology and dermatoglyphics. Forensic applications could include matching partial internal fingerprints from OCT scans when surface prints are smudged or incomplete.
\subsection{Conclusion}

This study demonstrates that 3D OCT data can be used to extract quantitative descriptors of sweat duct patterns and ridge curvature on the internal fingerprint surface as biomarkers for fingerprint recognition. These descriptors include spacing, curvature, and orientation. They capture geometric relationships that are stable, distinctive, and difficult to replicate artificially. Although the study has limitations regarding sample size, device specificity, and manual processing, the results suggest that Level 3 features from internal anatomy could significantly improve fingerprint recognition systems. By refining imaging techniques, analysis algorithms, and increasing dataset diversity, future work can help advance OCT-based fingerprint recognition toward deployment as a secure biometric modality.